Tato diplomová práce se zabývá vzory v uživatelské rozhraní, též známé jako temné vzory, které nutí uživatele dělat věci, nebo se rozhodovat jinak, než původně zamýšleli. Tato práce se zaměřuje na detekci temných vzorů použité webshopy na českém internetu a detekce probíhá ve velkém měřítku.

Práce vychází z již provedeného výzkumu z Princetonovi univerzity, který zkoumal temné vzory na anglických webshopech.

Byly vytvořeny několik nástrojů pro získání značného počtu webshopů. Nástroje z původního výzkumu byly upravené tak, aby mohly být použity pro český jazyk.

Těmito nástroji bylo získáno několik datasetů mapující webshopy na českém internetu a temné vzory na nich použité.

Bylo zjištěno, že temné vzory jsou na českých webshopech hojně využívány.