This thesis investigates patterns in user interfaces, also known as dark patterns, that force users to do things or make decisions differently than they originally intended. This thesis focuses on the detection of dark patterns used by webshops on the Czech Internet and the detection is done on a large scale.

This thesis builds on research already conducted at Princeton University that investigated dark patterns on English webshops.

Several tools were created to retrieve a significant number of webshops. Also the tools from the conducted research were modified to be applied to the Czech language.

These tools were used to obtain multiple datasets mapping webshops on the Czech Internet and the dark patterns used on them.

It was found that dark patterns are widely used on Czech webshops.