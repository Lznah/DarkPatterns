Dark patterns \cite{dark-patterns-brignull}\cite{dark-patterns-colin}\cite{the-year-dark-pattern-won}\cite{dark-patterns-at-scale} are ways of designing a user interface of websites, apps or any other computer system in a specific way to trick, confuse or coerce a user in doing unwanted actions like confirming to share more information than is needed to use the service, signing up for things that the user did not mean to, buying unwanted products and more .  

Typically, when the user reads a website or uses an app, he does not read all the words and makes quick assumptions\cite{dark-patterns-brignull}. Dark patterns then trick the user by hiding information of unpleasant truth. The user also trusts in his experience that he has gained from using other websites or apps and expects specific actions to happen or not to happen by using a similar pattern in the user interface. The user is tricked here by excepting this user interface behaviour, but in reality, it does something more or less than what the user expects\cite{the-year-dark-pattern-won}. Dark patterns are not only able to take advantage of the user not paying enough attention. Another dark pattern uses psychological methods to make users feel bad and guilty for not doing what the dark pattern wants them to do\cite{the-year-dark-pattern-won}.

Research into tricky user interface designs and deceptive practices has surpsingly a lot history, but it was neglected for many years. In 1999, Hanson and Kysar were the first who examined how companies abuse customers' cognitive limitations and profit from them\cite{kysar-douglas}. The rapid growth of the Internet and e-commerce increased more serious discussions and analyses of this topic. The term Dark Pattern itself was introduced by user interface expert Harry Brignull in 2010 to create a library of different types of dark patterns and to shame websites using them\cite{dark-patterns-brignull-about-us}. 

% https://www.darkpatterns.org/about-us

In March 2021, the state of California added new regulation that it now bans dark patterns that prevent users opting out the sale of their personal data\cite{california-bans-dark-patterns}. Therefore, the topic of dark patterns becomes more and more relevant.

% https://www.theverge.com/2021/3/16/22333506/california-bans-dark-patterns-opt-out-selling-data

In 2019, the group of scientist from Princeton University introduced and automated approach that enables experts to identify dark patterns used on websites at scale \cite{dark-patterns-at-scale}. 

% https://arxiv.org/pdf/1907.07032.pdf

The main goal of this thesis is build on top of their research to analyze prevalence of dark patterns on Czech webshops also described in\cite{dark-patterns-at-scale}. This thesis focuses on product pages and product purchase flow only, because these are the most promising pages, where all the buying happens. To fullfil this goal, several subgoals needs to be done:
\begin{itemize}
    \item An automated approach of gathering data from numerous Czech webshops at scale.
    \item These extracted data needs to be analysed in order to train a model that is able to detect dark patterns.
    \item Evaluate and describe results.
\end{itemize}

The thesis does not aim to study the prevalence of dynamic dark patters that display transients values over time.